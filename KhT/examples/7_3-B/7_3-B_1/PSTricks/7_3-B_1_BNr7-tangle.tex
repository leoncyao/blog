\documentclass{article}
\usepackage[crop=off]{auto-pst-pdf}
\usepackage{pst-node,rotating}
\renewcommand{\familydefault}{\sfdefault}
\begin{document}
\centering 
\psset{yunit=-1}\begin{pspicture}(-0.5,-0.5)(8.0,6.25)
\psset{linewidth=2.5pt}
\rput[c](3.75,0){\textbf{7\_3-B\_1\_BNr7}}
\rput[c](3.75,0.75){}

%%%%%%%%%%%%%%%%%%%
% [[['cap', 1]], [], [1, 2]]
\psbezier(1,1.75)(1,1.25)(2,1.25)(2,1.75)
\rput[c](7.0,1.25){\color{gray}cap1}
\psbezier(0,0.75)(0,1.25)(0,1.25)(0,1.75)

%%%%%%%%%%%%%%%%%%%
% [[['cap', 2]], [], [2, 3]]
\psbezier(2,2.75)(2,2.25)(3,2.25)(3,2.75)
\rput[c](7.0,2.25){\color{gray}cap2}
\psbezier(0,1.75)(0,2.25)(0,2.25)(0,2.75)
\psbezier(1,1.75)(1,2.25)(1,2.25)(1,2.75)
\psbezier(2,1.75)(2,2.25)(4,2.25)(4,2.75)
\psline[linecolor=lightgray](7.75,1.75)(-0.25,1.75)

%%%%%%%%%%%%%%%%%%%
% [[['cap', 2]], [], [2, 3]]
\psbezier(2,3.75)(2,3.25)(3,3.25)(3,3.75)
\rput[c](7.0,3.25){\color{gray}cap2}
\psbezier(0,2.75)(0,3.25)(0,3.25)(0,3.75)
\psbezier(1,2.75)(1,3.25)(1,3.25)(1,3.75)
\psbezier(2,2.75)(2,3.25)(4,3.25)(4,3.75)
\psbezier(3,2.75)(3,3.25)(5,3.25)(5,3.75)
\psbezier(4,2.75)(4,3.25)(6,3.25)(6,3.75)
\psline[linecolor=lightgray](7.75,2.75)(-0.25,2.75)

%%%%%%%%%%%%%%%%%%%
% [[['cup', 3]], [3, 4], []]
\psbezier(3,3.75)(3,4.25)(4,4.25)(4,3.75)
\rput[c](7.0,4.25){\color{gray}cup3}
\psbezier(0,3.75)(0,4.25)(0,4.25)(0,4.75)
\psbezier(1,3.75)(1,4.25)(1,4.25)(1,4.75)
\psbezier(2,3.75)(2,4.25)(2,4.25)(2,4.75)
\psbezier(5,3.75)(5,4.25)(3,4.25)(3,4.75)
\psbezier(6,3.75)(6,4.25)(4,4.25)(4,4.75)
\psline[linecolor=lightgray](7.75,3.75)(-0.25,3.75)

%%%%%%%%%%%%%%%%%%%
% [[['cup', 2]], [2, 3], []]
\psbezier(2,4.75)(2,5.25)(3,5.25)(3,4.75)
\rput[c](7.0,5.25){\color{gray}cup2}
\psbezier(0,4.75)(0,5.25)(0,5.25)(0,5.75)
\psbezier(1,4.75)(1,5.25)(1,5.25)(1,5.75)
\psbezier(4,4.75)(4,5.25)(2,5.25)(2,5.75)
\psline[linecolor=lightgray](7.75,4.75)(-0.25,4.75)
\end{pspicture}
\end{document}
